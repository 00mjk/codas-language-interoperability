\documentclass{beamer}

\mode<presentation>
{
  \usetheme{default}
  \usecolortheme{default}
  \usefonttheme{default}
  \setbeamertemplate{navigation symbols}{}
  \setbeamertemplate{caption}[numbered]
  \setbeamertemplate{footline}[page number]
  \setbeamercolor{frametitle}{fg=white}
  \setbeamercolor{footline}{fg=black}
} 

\usepackage[english]{babel}
\usepackage[utf8x]{inputenc}
\usepackage{tikz}
\usepackage{listings}
\usepackage{courier}
\usepackage{array}
\usepackage{bold-extra}
\usepackage{minted}

\xdefinecolor{darkblue}{rgb}{0.1,0.1,0.7}
\xdefinecolor{darkgreen}{rgb}{0,0.5,0}
\xdefinecolor{darkgrey}{rgb}{0.35,0.35,0.35}
\xdefinecolor{darkorange}{rgb}{0.8,0.5,0}
\xdefinecolor{darkred}{rgb}{0.7,0,0}
\xdefinecolor{dianablue}{rgb}{0.18,0.24,0.31}
\definecolor{commentgreen}{rgb}{0,0.6,0}
\definecolor{stringmauve}{rgb}{0.58,0,0.82}

\lstset{ %
  backgroundcolor=\color{white},      % choose the background color
  basicstyle=\ttfamily\small,         % size of fonts used for the code
  breaklines=true,                    % automatic line breaking only at whitespace
  captionpos=b,                       % sets the caption-position to bottom
  commentstyle=\color{commentgreen},  % comment style
  escapeinside={\%*}{*)},             % if you want to add LaTeX within your code
  keywordstyle=\color{blue},          % keyword style
  stringstyle=\color{stringmauve},    % string literal style
  showstringspaces=false,
  showlines=true
}

\lstdefinelanguage{scala}{
  morekeywords={abstract,case,catch,class,def,%
    do,else,extends,false,final,finally,%
    for,if,implicit,import,match,mixin,%
    new,null,object,override,package,%
    private,protected,requires,return,sealed,%
    super,this,throw,trait,true,try,%
    type,val,var,while,with,yield},
  otherkeywords={=>,<-,<\%,<:,>:,\#,@},
  sensitive=true,
  morecomment=[l]{//},
  morecomment=[n]{/*}{*/},
  morestring=[b]",
  morestring=[b]',
  morestring=[b]"""
}

\title[2017-07-13-language-interop]{Language interoperability}
\author{Jim Pivarski}
\institute{}
\date{Presentation: will redirect to notebooks}

\begin{document}

\logo{\pgfputat{\pgfxy(0.11, 8)}{\pgfbox[right,base]{\tikz{\filldraw[fill=dianablue, draw=none] (0 cm, 0 cm) rectangle (50 cm, 1 cm);}}}}

\begin{frame}
  \titlepage
\end{frame}

% Uncomment these lines for an automatically generated outline.
%\begin{frame}{Outline}
%  \tableofcontents
%\end{frame}

%%%%%%%%%%%%%%%%%%%%%%%%%%%%%%%%%%%%%%%%%%%%%%%%%%%%%%%

\begin{frame}{Organization of this tutorial}
\Large
\begin{enumerate}\setlength{\itemsep}{0.5 cm}
\item Overview of languages
\item Moving functions across languages
\item Moving data across languages
\item Projects: form pairs or small groups and work on one of seven problems
\end{enumerate}
\end{frame}

\begin{frame}{The Big Three: a basis set of features}
\vspace{0.4 cm}
\renewcommand{\arraystretch}{1.3}
\mbox{\hspace{-0.5 cm}\begin{tabular}{p{0.32\linewidth} p{0.34\linewidth} p{0.31\linewidth}}
C/C++ & Java/JVM & Python \\\hline
\mbox{compiles to machine} code & \mbox{compiles to VM;} \mbox{optimizes} at runtime & interpreted bytecode; runtime type-checks \\
\uncover<2->{\vspace{-0.9\baselineskip}\mbox{fastest; reproducible} performance & slow start and intermittent pauses, but relatively high throughput & \mbox{quick to start, but} terrible throughput} \\
\uncover<3->{\vspace{-0.9\baselineskip}\mbox{bare metal and high} abstractions coexist & \mbox{choice of language,} some high, some low & \mbox{generally only one} way to do things} \\
\uncover<4->{\vspace{-0.9\baselineskip}\mbox{manual memory} management (including smart pointers) & \mbox{generational garbage} \mbox{collectors, tunable to} improve throughput & \mbox{simple reference} \mbox{counting garbage} collector} \\
\uncover<5->{\vspace{-0.9\baselineskip}language superset of C, others through FFI & \mbox{hard to break out of} the VM & excellent interface to other languages}
\end{tabular}}
\end{frame}

\begin{frame}{Other languages, notable for being {\it different} from these}
\vspace{0.35 cm}
\begin{itemize}\setlength{\itemsep}{0.18 cm}
\item<1-> \textcolor{darkblue}{Rust:} like C++ without the history (cruft) and a typesystem that tracks ownership (inescapable smart pointers).
\item<2-> \textcolor{darkblue}{Go:} like C++ without the history (cruft) and an emphasis on modern concurrency (coroutines, mostly).
\item<3-> \textcolor{darkblue}{.NET:} multilingual ecosystem like the JVM that learned from Java's mistakes. Mostly Windows, though.
\item<4-> \textcolor{darkblue}{R:} like Python but specialized for statistical analysis; has a huge library (hint: use rpy2 to run R from Python).
\item<5-> \textcolor{darkblue}{Julia:} like Python but JIT-compiles each function before use; intended for serious number-crunching (hint: see Numba).
\item<6-> \textcolor{darkblue}{Javascript:} embedded in the GUI toolkit that absolutely everyone has installed on their laptops/phones/watches.
\item<7-> \textcolor{darkblue}{Haskell:} too pure and beautiful to touch. It can only be sullied by our unclean hands.
\end{itemize}
\end{frame}

\begin{frame}{Degrees of cohabitation}
\vspace{0.5 cm}
\begin{columns}[t]
\column{0.31\linewidth}
\textcolor{darkblue}{\underline{Same process}} \hfill \mbox{ }

\vspace{0.25 cm}
C/C++, Java, and Python can live in the same process, single or multiple threads, calling each other's functions (with care).

\vspace{0.25 cm}
Statically compiled or dynamically loaded.

\column{0.37\linewidth}
\begin{uncoverenv}<2->
\textcolor{darkblue}{\underline{Same machine}}

\vspace{0.25 cm}
It's safer and easier to use multiple processes, but now all communication is asynchronous; concurrency is an issue.

\vspace{0.25 cm}
\vspace{\baselineskip}
Shared memory (fast and dangerous), pipes, localhost sockets, or even intermediate files (don't, though).
\end{uncoverenv}

\column{0.31\linewidth}
\begin{uncoverenv}<3->
\textcolor{darkblue}{\underline{Same planet}}

\vspace{0.25 cm}
Communicating over a network allows for flexible scale-out, though at a cost of performance.

\vspace{0.25 cm}
\vspace{2\baselineskip}
Raw sockets, ZeroMQ sockets, HTTP, message queuing systems.
\end{uncoverenv}
\end{columns}
\end{frame}

\begin{frame}{Same process: static and dynamic}
\vspace{0.25 cm}
\begin{block}{Python}
Ships with a C API that is very good: nearly all high-performance libraries for Python use it.

\vspace{0.25 cm}
Some high-level wrappers on the C API: Cython, Boost::Python, SWIG, TPython (ROOT).

\vspace{0.25 cm}
Also has a builtin ctypes library to dynamically load .so files; a little-known gem. \hfill \textcolor{red}{[ctypes.ipynb]}
\end{block}

\begin{uncoverenv}<2->
\begin{block}{Java/JVM}
Ships with a Java Native Interface (JNI) to statically compile native functions into JAR files (Java shared libraries). Not often used in the Java community, prone to error.

\vspace{0.25 cm}
But the Java Native Access (JNA) library is a good way to dynamically load .so files (native shared library). \hfill \textcolor{red}{[jna.ipynb]}
\end{block}\end{uncoverenv}
\end{frame}

\begin{frame}{Shared object files: the least common denominator}





\end{frame}





\begin{frame}{Projects}
\begin{enumerate}
\item JNA has callbacks, pointers, and structs just like ctypes. Repeat the GSL root-finding example with JNA.


\end{enumerate}
\end{frame}

\end{document}
